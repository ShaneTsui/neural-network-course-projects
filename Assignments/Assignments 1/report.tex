\documentclass{article} % For LaTeX2e
\usepackage{nips15submit_e,times}
\usepackage{hyperref}
\usepackage{url}
\usepackage{mathtools}

%\documentstyle[nips14submit_09,times,art10]{article} % For LaTeX 2.09


\title{Report on Facial Expression Classification}


\author{
David S.~Hippocampus\thanks{ Use footnote for providing further information
about author (webpage, alternative address)---\emph{not} for acknowledging
funding agencies.} \\
Department of Computer Science\\
Cranberry-Lemon University\\
Pittsburgh, PA 15213 \\
\texttt{hippo@cs.cranberry-lemon.edu} \\
\And
Coauthor \\
Affiliation \\
Address \\
\texttt{email} \\
\AND
Coauthor \\
Affiliation \\
Address \\
\texttt{email} \\
\And
Coauthor \\
Affiliation \\
Address \\
\texttt{email} \\
\And
Coauthor \\
Affiliation \\
Address \\
\texttt{email} \\
(if needed)\\
}

% The \author macro works with any number of authors. There are two commands
% used to separate the names and addresses of multiple authors: \And and \AND.
%
% Using \And between authors leaves it to \LaTeX{} to determine where to break
% the lines. Using \AND forces a linebreak at that point. So, if \LaTeX{}
% puts 3 of 4 authors names on the first line, and the last on the second
% line, try using \AND instead of \And before the third author name.

\newcommand{\fix}{\marginpar{FIX}}
\newcommand{\new}{\marginpar{NEW}}

%\nipsfinalcopy % Uncomment for camera-ready version

\begin{document}


\maketitle

\begin{abstract}
The abstract paragraph should be indented 1/2~inch (3~picas) on both left and
right-hand margins. Use 10~point type, with a vertical spacing of 11~points.
The word \textbf{Abstract} must be centered, bold, and in point size 12. Two
line spaces precede the abstract. The abstract must be limited to one
paragraph.
\end{abstract}

\part{Solutions to individual problems}
\section{Problems from Bishop}

\subsection{Solution 1.1}

Let $\lambda=2$ and use (1.41) on the LHS of (1.42),

\begin{align*} 
    \prod_{i=1}^{d}\int_{-\infty}^{\infty}e^{-x_i^2}dx_i &= \prod_{i=1}^{d}\int_{-\infty}^{\infty}e^{-\frac{2}{2}x_i^2}dx_i \\
    &= \prod_{i=1}^{d}(\frac{2\pi}{2})^{\frac{1}{2}} \\
    &= \pi^{\frac{d}{2}}
\end{align*} 

Compare the result with RHS of (1.42),

\begin{align*} 
    S_d &= \frac{\pi^{\frac{d}{2}}}{\int_{0}^{\infty}e^{-r^2}r^{d-1}dr} \\
    &= \frac{2 \pi^{\frac{d}{2}}}{\int_{0}^{\infty}e^{-r^2}r^{d-2}2r dr} \\
    &= \frac{2 \pi^{\frac{d}{2}}}{\int_{0}^{\infty}e^{-r^2}r^{d-2}dr^2}
\end{align*} 

We substitute $r^2$ with $u$. Since $r>0$, $r=u^{\frac{1}{2}}$.

\begin{align*} 
    S_d &= \frac{2 \pi^{\frac{d}{2}}}{\int_{0}^{\infty}e^{-r^2}r^{d-2}dr^2} \\
        &= \frac{2 \pi^{\frac{d}{2}}}{\int_{0}^{\infty}e^{-u}u^{\frac{d}{2}-1}du} \\
        &= \frac{2 \pi^{\frac{d}{2}}}{\Gamma(\frac{d}{2})}
\end{align*} 

Thus, we proved the correctness of (1.43).

Then, we verify $S_d$ is the surface area of the unit sphere in d dimensions, which should be $S_2=2\pi$ and $S_4=4\pi$.

% $$sphere_d = \frac{4}{3} \pi r^3$$

\begin{align*}
    S_2 &= \frac{2\pi^{\frac{2}{2}}}{\Gamma(1)} \\
    &= \frac{2\pi}{1} \\
    &= 2\pi
\end{align*} 

\begin{align*}
    S_3 &= \frac{2\pi^{\frac{3}{2}}}{\Gamma(\frac{3}{2})} \\
    &= \frac{2\pi^{\frac{3}{2}}}{\frac{\pi^{\frac{1}{2}}}{2}} \\
    &= 4\pi
\end{align*} 

Thus, we verified the correctness of (1.43).

\subsection{Solution 1.2}

Since
\begin{align*}
    V_d = \frac{\pi^{d/2}}{\Gamma(\frac{d}{2} + 1)} a^d
\end{align*} 
and
\begin{align*}
    \Gamma(x + 1) = x\Gamma(x)
\end{align*} 
so
\begin{align*}
    V_d &= \frac{\pi^{d/2}}{\frac{d}{2}\Gamma(\frac{d}{2})} a^d \\
    &= \frac{2\pi^{d/2}}{\Gamma(\frac{d}{2})} \frac{a^d}{d} \\
    &= \frac{S_d a^d}{d}
\end{align*} 
Thus, (1.45) is proved.

The volume of a d dimensional hypercube with edges a is

\begin{align*}
    V_d &= a^d \\
\end{align*}

So, in d dimension, given a hypercube with edge 2a and hypersphere with radius a, we have
\begin{align*}
    \frac{volume\ of\ a\ sphere}{volume\ of\ a\ cube} &= \frac{\frac{S_d a^d}{d}}{(2a)^d} \\
    &= \frac{S_d}{d2^d} \\
    &= \frac{\pi^{\frac{d}{2}}}{d2^{d-1}\Gamma(\frac{d}{2})}
\end{align*} 
Thus, we've proved (1.46).

Using (1.46) and (1.47),
\begin{align*}
    \lim_{d \to \infty} \frac{volume\ of\ a\ sphere}{volume\ of\ a\ cube} &= \lim_{d \to \infty} \frac{\pi^{\frac{d}{2}}}{d2^{d-1}\Gamma(\frac{d}{2})} \\
    &= \lim_{d \to \infty} \frac{\pi^{\frac{d}{2}}}{d2^{d-1}\Gamma((\frac{d}{2}-1)+1)} \\
    &= \lim_{d \to \infty} \frac{\pi^{\frac{d}{2}}}{d2^{d-1}(2\pi)^{\frac{1}{2}}e^{-(\frac{d}{2}-1)}(\frac{d}{2}-1)^{\frac{d+1}{2}}} \\
\end{align*}

Let $x = \frac{d}{2} - 1$,
\begin{align*}
    \lim_{d \to \infty} \frac{volume\ of\ a\ sphere}{volume\ of\ a\ cube} &= \lim_{d \to \infty} \frac{\pi^{\frac{d}{2}}}{d2^{d-1}(2\pi)^{\frac{1}{2}}e^{-(\frac{d}{2}-1)}(\frac{d}{2}-1)^{\frac{d+1}{2}}} \\
    &= \lim_{x \to \infty} \frac{\pi^{x+\frac{1}{2}}}{2^{\frac{5}{2}}(x+1)x^{\frac{3}{2}}(\frac{4x}{e})^x} \\
    &= \lim_{x \to \infty} [\frac{\pi^{\frac{1}{2}}}{2^{\frac{5}{2}}(x+1)x^{\frac{3}{2}}} (\frac{e\pi}{4x})^x] \\
    &= \lim_{x \to \infty} [\frac{\pi^{\frac{1}{2}}}{2^{\frac{5}{2}}(x+1)x^{\frac{3}{2}}} e^{x \ln \left(\frac{e\pi}{4x} \right)}] \\
    &= (\frac{\pi}{32})^{\frac{1}{2}} \lim_{x \to \infty} \frac{(e^{\frac{e\pi}{4}})^{-(\frac{4x}{e\pi}) \ln \left(\frac{4x}{e\pi} \right)}}{(x+1)x^{\frac{3}{2}}} \\
\end{align*} 

For numerator, let $y = \frac{4x}{e\pi}$, then we have
\begin{align*}
    &= \lim_{x \to \infty} (e^{\frac{e\pi}{4}})^{-(\frac{4x}{e\pi}) \ln \left(\frac{4x}{e\pi} \right)}\\
    &= \lim_{y \to \infty} (e^{\frac{e\pi}{4}})^{-y \ln y}\\
    &= (e^{\frac{e\pi}{4}})^{\lim_{y \to \infty} -y \ln y} \\
    &= 0.
\end{align*} 

For denominator,
\begin{align*}
    &= \lim_{d \to \infty} (x+1)x^{\frac{3}{2}}\\
    &= \infty. \\
\end{align*} 

Combining the limits above, it's obvious that
\begin{align*}
    \lim_{d \to \infty} \frac{volume\ of\ a\ sphere}{volume\ of\ a\ cube} &= 0.
\end{align*} 

Given a d dimensional hypercube with edge a, the distance from center to one corner is $\frac{1}{2}a\sqrt{d}$, and the distance from
the centre to one face is $\frac{1}{2}a$. Thus the ratio is
\begin{align*}
    \frac{distance\ from\ center\ to\ one\ corner}{distance\ from\ the\ centre\ to\ one\ face} = \sqrt{d}.
\end{align*} 

And it's obvious that
\begin{align*}
    \lim_{d \to \infty} \sqrt{d} = \infty.
\end{align*} 

\subsubsection{Solution 1.3}

Using (1.45), we have
\begin{align*}
    f &= (\frac{S_d a^d}{d} - \frac{S_d (a-\epsilon)^d}{d} ) / \frac{S_d a^d}{d} \\
    &=  1 - (1 - \frac{\epsilon}{a})^d. \\
\end{align*}

Since $0 < \epsilon < a$, it's obvious that $0 < 1 - \frac{\epsilon}{a} < 1$.
So
\begin{align*}
    \lim_{d \to \infty} [1 - (1 - \frac{\epsilon}{a}) ^d] &= 1 - \lim_{d \to \infty}(1 - \frac{\epsilon}{a}) ^d \\
    &= 1 - 0 \\
    &= 1
\end{align*} 

Then, we evaluate the value of f as following.

\begin{table}[h]
\caption{Evaluation of f}
\label{evaluation-of-f}
\begin{center}
\begin{tabular}{lll}
\multicolumn{1}{c}{\bf $\epsilon/a$}  &\multicolumn{1}{c}{\bf d} &\multicolumn{1}{c}{\bf f}
\\ \hline \\
0.01 & 2 & 0.01990000000000003 \\
0.01 & 10 & 0.09561792499119559 \\
0.01 & 1000 & 0.9999568287525893 \\
0.02 & 2 & 0.03960000000000008 \\
0.02 & 10 & 0.1829271931124533 \\
0.02 & 1000 & 0.9999999983170327 \\
\end{tabular}
\end{center}
\end{table}

\begin{align*}
\end{align*} 
\begin{align*}
\end{align*} 
\begin{align*}
\end{align*} 
\begin{align*}
\end{align*} 
\begin{align*}
\end{align*} 
\begin{align*}
\end{align*} 
\begin{align*}
\end{align*} 
\begin{align*}
\end{align*} 


\end{document}